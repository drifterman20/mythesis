\mychapter{0}{\textit{Abstract}}

\begin{comment}
    
    
%The objective of this project is to create an Artificial Intelligence-based system capable of identifying automatically medical labels and extracting their text content. We started by using 2 methods with different approaches (Tesseract-based approach And VisionAI-based approach ) to have a comparison at the end of the project between those two approaches. The major distinction between the 2 approaches is that the VisionAI-based approach uses an Intelligent detection of the medical label whereas the Tesseract-based approach uses a manual selection of the corners of the medical label. In the initial method (Tesseract-based approach) entails some traditional detection and segmentation using classic image processing techniques like the horizontal and vertical histogram analysis are employed for detection and segmentation of lines and parts of the medical label, alongside the utilization of PyTesseract, an OCR tool, to identify and extract text from the medical labels. Regarding the second method, we were able to create a custom-trained medical label detection model that displayed precise and reliable outcomes, alongside utilizing one of the bezt OCR engines called VisionAI. Ultimately, VisionAI-based approach  has demonstrated a great performance of detection of medical labels and extracting their content with an accuracy of about 95.79\%.

In the field of pharmacology, medication labels play a crucial role in providing essential information about the medicines being sold or distributed, including details such as lot numbers, fabrication and expiration dates, and other pertinent information. However, the absence of a standardized model for medical labels approved by the Minister of Health has resulted in a lack of automatic recognition systems for medical labels in Algeria. In this work, we propose an Artificial Intelligence (AI)-based system capable of automatically identifying medical labels, extracting their content.%, and classifying them as refundable or non-refundable medications.

The study introduces two approaches: the Tesseract-based approach and the VisionAI-based approach. The Tesseract-based approach involves traditional detection and segmentation techniques using classic image processing methods, such as horizontal and vertical histogram analysis, for the identification and segmentation of lines and sections within the medical labels. Additionally, PyTesseract, an Optical Character Recognition (OCR) tool, is utilized to extract text from the medical labels. On the other hand, the VisionAI-based approach incorporates a custom-trained medical label detection and segmentation model, which demonstrates precise and reliable outcomes. Furthermore, the approach leverages VisionAI, a highly accurate OCR engine, for text extraction from medical labels.

Experimental evaluations demonstrate the superior performance of the VisionAI-based approach, achieving a remarkable accuracy of approximately 95.79\% in the detection and extraction of their content. This research presents a promising solution to the lack of automated recognition systems for medical labels in Algeria, providing a foundation for the implementation of efficient and reliable tools for medication management and classification. %Further research is warranted to enhance the system's capabilities and expand its applicability within the healthcare domain.
\end{comment}
In the context of pharmacology and healthcare, medical labels contain essential information such as batch numbers, manufacturing and expiration dates, dosage instructions, and pricing. However, due to the absence of a standardized medical label format approved by Algerian health authorities, the development of automated systems for medical label recognition remains a significant challenge.
This work proposes an Artificial Intelligence (AI)-based system to automatically recognizing and extracting text from medical labels. To address the problem, we constructed a dedicated dataset composed of over 2,500 labeled medical label images. To this end we propose three different OCR engines: Tesseract, EasyOCR, and TrOCR and evaluate them on the mentioned dataset. Additionally, the study is expanded with fine-tuned models and further hyperparameter optimization (HPO) to improve the models' performance in a domain-specific context.
The HPO focuses on the application of two hyperparameter optimization methods— Optuna Configurator and Genetic Algorithms (GA)—to enhance the proposed OCR model's performance. %Optuna relies on a Bayesian optimization strategy, while GA employs evolutionary mechanisms such as selection, crossover, and mutation. 
The obtained results shows that Genetic Algorithms consistently outperformed Optuna optimization technique across all three OCR models.
%Among the models tested, TrOCR—when optimized using GA with Tournament selection and TwoPoint crossover—achieved the lowest Character Error Rate (CER) of 0.0400. EasyOCR also achieved strong results with a best CER of 0.0600, while Tesseract improved to 0.0627.
These findings demonstrate that TrOCR, paired with evolutionary optimization, is the most effective approach for automated text recognition in Algerian medical labels.
This project presents a robust solution for medical label text extraction, paving the way for intelligent systems capable of assisting medication classification and improving pharmaceutical data management in Algeria.
\vspace{1cm}

%\noindent\rule[2pt]{\textwidth}{0.5pt}

{\textbf{Keywords :}}
Artificial Intelligence, Medical labels, Text recognition, Detection, Segmentation, Classification, OCR, EasyOCR, TrOCR, Tesseract, Hyperparameter Optimization, Genetic Algorithms, Optuna.
\\